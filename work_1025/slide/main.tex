\documentclass[dvipdfmx]{beamer}
\usetheme{metropolis}           % Use metropolis theme
\usepackage{float}
\usepackage{ascmac}
\usepackage{fancybox}
\usepackage{tikz}
\usetikzlibrary {arrows.meta}
\usetikzlibrary {bending}
\usepackage{listings,jvlisting} %日本語のコメントアウトをする場合jvlisting(もしくはjlisting)が必要
%ここからソースコードの表示に関する設定
\lstset{
  basicstyle={\ttfamily},
  identifierstyle={\small},
  commentstyle={\smallitshape},
  keywordstyle={\small\bfseries},
  ndkeywordstyle={\small},
  stringstyle={\small\ttfamily},
  frame={tb},
  breaklines=true,
  columns=[l]{fullflexible},
  numbers=left,
  xrightmargin=0zw,
  xleftmargin=3zw,
  numberstyle={\scriptsize},
  stepnumber=1,
  numbersep=1zw,
  lineskip=-0.5ex
}

\title{Progress}
\date{\today}
\author{Mizuno Yasuaki}
%\institute{Centre for Modern Beamer Themes}
\begin{document}
  \maketitle

  \begin{frame}{目次}
    \begin{enumerate}
      \item k分割交差検証
    \end{enumerate}
  \end{frame}
  
  \begin{frame}{k分割交差検証}
    \begin{itemize}
        \item 訓練データを同じサイズのk個のサブセットに分ける
        \item (k - 1)個のサブセットで訓練し、残りのサブセットで評価する
        \item 最終的にk個のスコアの平均
    \end{itemize}
  \end{frame}

  \begin{frame}
    \begin{figure}
      \centering
      \begin{tikzpicture}
        % 一列目
        \draw [fill=yellow, rounded corners=10pt] (0, 3)rectangle(2, 4) (1, 3.5)node{検証};
        \draw [rounded corners=10pt] (2.5, 3)rectangle(4.5, 4) (3.5, 3.5)node{訓練};
        \draw [rounded corners=10pt] (5, 3)rectangle(7, 4) (6, 3.5)node{訓練};
        \draw [->] (7.5, 3.5)--(8, 3.5) node[right]{検証スコア①};
        % 二列目
        \draw [rounded corners=10pt] (0, 1.5)rectangle(2, 2.5) (1, 2)node{訓練};
        \draw [fill=yellow, rounded corners=10pt] (2.5, 1.5)rectangle(4.5, 2.5) (3.5, 2)node{検証};
        \draw [rounded corners=10pt] (5, 1.5)rectangle(7, 2.5) (6, 2)node{訓練};
        \draw [->] (7.5, 2)--(8, 2) node[right]{検証スコア②};
        % 三列目
        \draw [rounded corners=10pt] (0, 0)rectangle(2, 1) (1, .5)node{訓練};
        \draw [rounded corners=10pt] (2.5, 0)rectangle(4.5, 1) (3.5, .5)node{訓練};
        \draw [fill=yellow, rounded corners=10pt] (5, 0)rectangle(7, 1) (6, .5)node{検証};
        \draw [->] (7.5, .5)--(8, .5) node[right]{検証スコア③};
      \end{tikzpicture}
      \caption{3分割交差検証}
    \end{figure}
  \end{frame}

  \begin{frame}{k分割交差検証結果}
    \begin{table}[H]
      \centering
      \caption{4分割交差検証}
      \begin{tabular}{lll}
        \hline
        i & loss & accuracy \\
        \hline \hline
        0 & \\
        1 & \\
        2 & \\
        3 & \\
        \hline
      \end{tabular}
    \end{table}
  \end{frame}

\end{document}


