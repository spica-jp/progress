\documentclass[dvipdfmx]{beamer}
\usetheme{metropolis}           % Use metropolis theme
\usepackage{float}
\usepackage{listings,jvlisting} %日本語のコメントアウトをする場合jvlisting(もしくはjlisting)が必要
%ここからソースコードの表示に関する設定
\lstset{
  basicstyle={\ttfamily},
  identifierstyle={\small},
  commentstyle={\smallitshape},
  keywordstyle={\small\bfseries},
  ndkeywordstyle={\small},
  stringstyle={\small\ttfamily},
  frame={tb},
  breaklines=true,
  columns=[l]{fullflexible},
  numbers=left,
  xrightmargin=0zw,
  xleftmargin=3zw,
  numberstyle={\scriptsize},
  stepnumber=1,
  numbersep=1zw,
  lineskip=-0.5ex
}

\title{Progress}
\date{\today}
\author{Mizuno Yasuaki}
%\institute{Centre for Modern Beamer Themes}
\begin{document}
  % 表紙の作成
  \maketitle
  
  \begin{frame}{ファミリー数削減}
    前回はファミリー数を百前後で学習を行って、テストデータの精度を表\ref{tab:accuracy}に示す。
    ファミリー数を十に減らして、様々なニューラルネットワークのモデルを試す。
    \begin{table}
      \caption{accuracy}
      \label{tab:accuracy}
      \centering
      \begin{tabular}{ll}
        \hline
        model & accuracy \\
        \hline \hline
        FNN & 0.2889 \\
        CNN & 0.8783 \\
        \hline
      \end{tabular}
    \end{table}
  \end{frame}

  \begin{frame}{削減データの概要}
    ファミリー数を十に減らしたときの結果を表\ref{tab:less_data}に示す。

    \begin{table}
      \caption{less\_data}
      \label{tab:less_data}
      \centering
      \begin{tabular}{ll}
        \hline
        data\_name & data\_size \\
        \hline \hline
        train\_data & 66704 \\
        test\_data & 592 \\
        \hline
      \end{tabular}
    \end{table}
  \end{frame}

  \begin{frame}{使用したニューラルネットワーク}

  \end{frame}

  \begin{frame}{削減データを用いた学習結果}
    \begin{table}
      
    \end{table}
  \end{frame}

\end{document}


