\documentclass[leno,xcolor=dvipsnames]{beamer}
\usetheme[
    block=fill,         % ブロックに背景をつける
    progressbar=foot,   % 各スライドの下にプログレスバー
    numbering=fraction  % 合計ページ数を表示
]{metropolis}           % Use metropolis theme

\usepackage{luatexja}% 日本語したい
\usepackage[ipaex]{luatexja-preset}% IPAexフォントしたい
\renewcommand{\kanjifamilydefault}{\gtdefault}% 既定をゴシック体に

\usepackage{float}
\usepackage{booktabs}
\usepackage{ascmac}
\usepackage{fancybox}
\usepackage{amsmath}
\usepackage{mathtools}
\usepackage{siunitx}
\usepackage{tikz}
\usetikzlibrary {arrows.meta}
\usetikzlibrary {bending}
\usepackage{listings}
\lstset{
    frame=single,
    basicstyle=\tiny\ttfamily,
    tabsize=4
}

\title{進捗報告}
\date{\today}
\author{水野泰旭}
\institute{弘前大学理工学部電子情報工学科4年}
\subject{}
\begin{document}
  \maketitle

\begin{frame}
    \section{k分割交差検証(比率を考慮)}
\end{frame}

\begin{frame}{手順}
    \begin{enumerate}
        \item クラスごとに画像データとラベルをそれぞれ分ける
        \item クラスごとに画像データとラベルをk個に分ける
        \item 以下の操作をk回繰り返す
        \begin{enumerate}
            \item それぞれのクラスにおいてk個に分けた画像データとラベルデータを検証データ用に一つ選ぶ
            \item 検証データ以外を訓練データとしてまとめる
            \item 機械学習を行う
        \end{enumerate} 
        \item すべての学習終了後、ロスと正解率の平均を計算する
    \end{enumerate}
\end{frame}

\begin{frame}{10分割交差検証}
    \begin{table}[H]
        \centering  
        \begin{tabular}{rSS}
            \toprule    
            & ロス & 正解率 \\
            \midrule
            1回目 & 0.1895 & 0.9520 \\ 
            2回目 & 0.1073 & 0.9734 \\ 
            3回目 & 0.1510 & 0.9621 \\ 
            4回目 & 0.1698 & 0.9633 \\ 
            5回目 & 0.1287 & 0.9570 \\ 
            6回目 & 0.1278 & 0.9621 \\
            7回目 & 0.1953 & 0.9621 \\
            8回目 & 0.2565 & 0.9381 \\
            9回目 & 0.1493 & 0.9608 \\
            10回目 & 0.1506 & 0.9671 \\
            \midrule
            平均 & 0.1626 & 0.9597 \\
            \bottomrule
        \end{tabular}
    \end{table}
\end{frame}

\end{document}


