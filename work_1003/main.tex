\documentclass[dvipdfmx]{beamer}
\usetheme{metropolis}           % Use metropolis theme
\usepackage{metropolis_framesubtitle}
\usepackage{float}
\usepackage{listings,jvlisting} %日本語のコメントアウトをする場合jvlisting(もしくはjlisting)が必要
%ここからソースコードの表示に関する設定
\lstset{
  basicstyle={\ttfamily},
  identifierstyle={\small},
  commentstyle={\smallitshape},
  keywordstyle={\small\bfseries},
  ndkeywordstyle={\small},
  stringstyle={\small\ttfamily},
  frame={tb},
  breaklines=true,
  columns=[l]{fullflexible},
  numbers=left,
  xrightmargin=0zw,
  xleftmargin=3zw,
  numberstyle={\scriptsize},
  stepnumber=1,
  numbersep=1zw,
  lineskip=-0.5ex
}

\title{Progress}
\date{\today}
\author{Mizuno Yasuaki}
%\institute{Centre for Modern Beamer Themes}
\begin{document}
  \maketitle
  
  \begin{frame}{目次}
    \begin{enumerate}
      \item 研究内容
    \end{enumerate} 
  \end{frame}

  \begin{frame}{研究内容}{概要}
    『アミノ酸配列の画像を用いた機械学習によるタンパク質ファミリー
    \footnote{
      \begin{itemize}
        \item 進化的類縁関係を持つタンパク質のグループ
        \item 共通の機能を持つタンパク質のグループ
      \end{itemize}
    }
    分類』\\
    $\rightarrow$ アミノ酸配列の画像からタンパク質ファミリーを予測する
  \end{frame}
\end{document}


