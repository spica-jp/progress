\documentclass[leno,xcolor=dvipsnames]{beamer}
\usetheme[
    block=fill, % ブロックに背景をつける
    progressbar=foot, % 各スライドの下にプログレスバー
    numbering=fraction % 合計ページ数を表示
]{metropolis}

\usepackage{luatexja}% 日本語したい
\usepackage[ipaex]{luatexja-preset}% IPAexフォントしたい
\renewcommand{\kanjifamilydefault}{\gtdefault}% 既定をゴシック体に

\usepackage{float}
\usepackage{booktabs}
\usepackage{ascmac}
\usepackage{fancybox}
\usepackage{amsmath}
\usepackage{mathtools}
\usepackage{siunitx}
\usepackage{tikz}
\usepackage{url}
\usetikzlibrary {arrows.meta}
\usetikzlibrary {bending}
\usepackage{listings}
\lstset{
    frame=single,
    basicstyle=\tiny\ttfamily,
    tabsize=4
}
\usepackage{booktabs}
\usepackage{multirow}

\title{進捗報告}
\date{\today}
\author{水野泰旭}
\institute{弘前大学理工学部電子情報工学科4年}
\subject{}
\begin{document}
  \maketitle

  \begin{frame}
    \section{F値}
  \end{frame}

  \begin{frame}{F値とは}
    \begin{block}{\href{https://atmarkit.itmedia.co.jp/ait/articles/2210/24/news034.html}{定義}}
      統計学・機械が空襲におけるF値もしくはFスコアとは基本的に二値分類のタスク評価指標の一つで、
      適合率と再現率のトレードオフ関係に着目し、2つの値を調和平均した値のことである。
      0.0 \textasciitilde 1.0の範囲の値になり、1.0に近づくほどよい。1.0に近いことは適合率と再現率の両方が
      同時にできるだけ高いことを意味するので、最も効率よくバランスの取れた機械学習モデルと言える。
    \end{block}
  \end{frame}

  \begin{frame}{二値分類のタスクにおけるF値}
    \begin{table}[H]
      \centering
      \begin{tabular}{cccc}
        \toprule
        \multicolumn{2}{c}{F値\footnote{適合率と再現率の調和平均}} & \multicolumn{2}{c}{予測値} \\ \cline{3-4}
        & & 陽性 & 陰性 \\
        \midrule
        \multirow{2}{*}{正解率} & 陽性 & TP:True Positive & FN:False Negative \\
        & 陰性 & FP:False Positive & TN:True Negative \\
        \bottomrule
      \end{tabular}
    \end{table}
    \begin{itemize}
      \item 適合率 \mbox{} 陽性の予測がより正確になっていること \[ \mathrm{Precision} = \frac{TP}{TP + FP} \]
      \item 再現率 \mbox{} 陽性の予測での取りこぼしを少なくしたい \[ \mathrm{Recall} = \frac{TP}{TP + FN} \]
    \end{itemize}
  \end{frame}

  \begin{frame}{二値分類のタスクにおけるF値}
    \begin{eqnarray*}
      F_1score 
      &=& \frac{2}{\frac{1}{Precision} + \frac{1}{Recall}} \\
      &=& \frac{2}{\frac{\mathrm{TP} + \mathrm{FP}}{\mathrm{TP}} + \frac{\mathrm{TP} + \mathrm{FN}}{\mathrm{TP}}} \\
      &=& \frac{2\mathrm{TP}}{2\mathrm{TP} + \mathrm{FP} + \mathrm{FN}}
    \end{eqnarray*}
  \end{frame}

  \begin{frame}{\href{https://qiita.com/jyori112/items/110596b4f04e4e1a3c9b}{多クラス分類タスクにおけるF値}}
    クラスを一つを選びPositiveとし、それ以外をNegativeとして評価値を計算したあとそれらの平均を取ることで
    全体の評価値を計算する。このときの平均の取り方は以下の二種類ある。
    \begin{itemize}
      \item macro平均 
      \item micro平均 
    \end{itemize}
  \end{frame}

  \begin{frame}{macro平均}
    $class \in \{A, B, C\}$ において、三つのクラスの分類タスクを考える。
    \begin{eqnarray*}
      \mathrm{Accuracy}_{\mathrm{class}} &=& \frac{\mathrm{TP}_{\mathrm{class}} + \mathrm{TN}_{\mathrm{class}}}{\mathrm{TP}_{\mathrm{class}} + \mathrm{TN}_{\mathrm{class}} + \mathrm{FP}_{\mathrm{class}} + \mathrm{FN}_{\mathrm{class}}} \\
      \mathrm{Recall}_{\mathrm{class}} &=& \frac{\mathrm{TP}_{\mathrm{class}}}{\mathrm{TP}_{\mathrm{class}} + \mathrm{FN}_{\mathrm{class}}} \\
      \mathrm{Precision}_{\mathrm{class}} &=& \frac{\mathrm{TP}_{\mathrm{class}}}{\mathrm{TP}_{\mathrm{class}} + \mathrm{FP}_{\mathrm{class}}}
    \end{eqnarray*}
  \end{frame}

  \begin{frame}{micro平均}
    各クラスのexample数が大きく異なる場合に、macro平均では実際の精度が計算できない場合がある。そこでmicro平均を用いる。
    \begin{equation*}
      \mathrm{micro\_Accuracy} = \frac{\text{各クラスにおける正解した個数の和}}{\text{各クラスにおける個数の和}}
    \end{equation*}
    micro平均はAccuracyに限らず、PrecisionとRecall、F値に対しても計算できるが、すべてAccuracyと同じ値になる。
  \end{frame}

  \begin{frame}{これからやること}
    \begin{itemize}
      \item 先行研究の論文を読んで自分の研究と比較する
      \begin{itemize}
        \item DeepFam
        \item DeepHiFam
      \end{itemize}
    \end{itemize}
  \end{frame}

\end{document}


